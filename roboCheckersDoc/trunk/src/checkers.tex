\chapter{Il package Checkers}
Il \emph{package} \texttt{it.polito.Checkers} racchiude un insieme di
classi in grado di gestire una partita a Dama.\\ 
Il motore del gioco � racchiuso nella classe \texttt{Engine}, mentre le altre
classi fungono da interfaccia verso il resto del progetto.\\
Il motivo di questa organizzazione � dato dal fatto che tutti i metodi di
\texttt{Engine} provengono da una Java applet sviluppata da un gruppo di studenti
del \emph{California Institute of Technology} (http://www.cs.caltech.edu).
Essendo questo codice poco orientato ad oggetti e per la quasi totalit�
costituito da metodi statici, � stato racchiuso in una sola classe i cui metodi
vengono richiamati dagli oggetti \emph{wrapper} che la circondano.\\
La scelta di impiegare codice gi� pronto in questa parte � stata
mutuata essenzialmente dai limiti temporali (anche se non nascondiamo che in pi�
occasioni avremmo preferito riscriverlo da zero!) nonch� dal fatto che
implementare l'intera intelligenza per il gioco della dama avrebbe esulato dalla
natura del progetto stesso.
\section{Engine}
Il cuore della classe � costituito dai metodi \texttt{MiniMax()},
\texttt{Evaluate()} e \texttt{generate\_moves()}. Tutti gli altri metodi
servono a fornire supporto a questi tre.\\
Tra essi \texttt{generate\_moves()} � quello pi� complicato, perch� si occupa di
creare il vettore delle mosse possibili per un determinato giocatore, data una
qualsiasi configurazione (purch� valida) delle pedine sulla scacchiera.
Pertanto si serve di una serie di altre funzioni che, sostanzialmente,
determinano quali mosse, tra tutte quelle possibili, sono effettivamente valide
secondo le regole della Dama.
\subsection{MiniMax}
Nella teoria dei giochi un algoritmo si definisce di tipo \emph{MiniMax}, se �
finalizzato a \emph{mini}mizzare la \emph{massima} perdita possibile.\\
In un gioco a turni, come la Dama, l'algoritmo si esprime in modo ricorsivo
come segue:
\lstset{tabsize=2}
\begin{lstlisting}
function minimax(node, depth)
	IF is_terminal(node) OR depth = 0
		return Evaluate(Nodo)
	IF turn = opponent
		alpha := +inf
		FOREACH child of node
			alpha := min(alpha, minimax(child, depth-1))
	ELSE {our turn}
		alpha := -inf
		FOREACH child of node
			alpha := max(alpha, minimax(child, depth-1))
	return alpha
\end{lstlisting}
Dato l'albero delle mosse possibili e data una profondit� massima di
ricorsione, si simula l'esecuzione della mossa che minimizza il valore della
migliore posizione raggiungibile dall'altro giocatore. Quindi l'algoritmo assegna
un valore ad ogni mossa legale, proporzionale a quanto essa diminuisce il valore della
posizione per l'altro giocatore.\\
La funzione \texttt{Evaluate} � fondamentale per valutare, la
bont� di quel determinato stato del gioco (i.e. quanto sia desiderabile per il
dato giocatore raggiungere quella posizione). Se un nodo � terminale (mossa
vincente) la funzione di valutazione deve ritornare $\pm\infty$ a seconda che il
turno corrente sia rispettivamente del primo o del secondo giocatore.
Altrimenti significa che si � raggiunto il limite di profondit� prefissato, in questo caso
\texttt{Evaluate} effettuer� una stima delle pedine in gioco assegnando un peso
ad ognuna di esse in funzione della loro posizione e del loro valore (ad
esempio, una dama vale di pi� di una normale pedina).\\
Ovviamente, se si fissasse una profondit� di ricorsione pari a infinito, non
sarebbe necessaria la stima euristica data da \texttt{Evaluate} e si
determinerebbe ad ogni passo una mossa ottima; � altrettanto chiaro tuttavia che
questa strategia � applicabile unicamente in giochi estremamente banali (ad
esempio il Tris), in quanto il numero di nodi da valutare cresce
esponenzialmente con la profondit� di ricerca ed � pari $\overline{m}^d$ (dove
$\overline{m}$ � il numero medio di mosse possibili e $d$ � la profondit�)
determinando una complessit� NP completa.

\section{Square}
Rappresenta una casella sulla scacchiera. Mantiene semplicemente due attributi
interi che rappresentano le coordinate della casella.\\
I metodi contenuti in \texttt{Engine} non usano questo tipo di astrazione, ma
tipicamente mantengono separate le coordinate; la conversione in oggetti
\texttt{Square} viene eseguita dalle classi \texttt{Move} e \texttt{Board}

\section{Move e MoveCollection}
\section{Board}