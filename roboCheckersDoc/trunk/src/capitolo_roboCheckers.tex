\chapter{Il package roboCheckers}
Il \emph{package} \texttt{it.polito.roboCheckers} contiene le classi per la gestione della partita e dei giocatori.\\
Sono presenti un'interfaccia Player successivamente implementata in \texttt{Human\-Player} e \texttt{ComputerPlayer} e la classe \texttt{Game} che effettua la partita vera e propria.

\section{L'interfaccia Player}
Questa interfaccia � usata per uniformare il giocatore robotico, il quale deve pensare ed effettuare la mossa, con il giocatore umano, di cui, invece, deve capirne la mossa effettuata. \\
Il metodo che esegue tutto questo �:
\begin{lstlisting}
/** Ritorna la mossa effettuata dal giocatore */
Move makeMove(final Board board) throws CantMoveException, IllegalMoveException, NotCalibratedException;
\end{lstlisting}
Come si pu� notare dalle eccezioni lanciabili, il metodo permette di riconoscere il caso in cui non ci siano pi� mosse disponibili sia da parte del robot che dell'umano e il caso in cui venga fatta una mossa non valida da parte dell'umano. Ovviamente tutto ci� non pu� funzionare se non � prima avvenuta la calibrazione (NotCalibratedException).

\subsection{ComputerPlayer}
Il suo metodo principale (\texttt{makeMove()}) si pu� sostanzialmente suddividere in due parti. Inizialmente viene scelta tra le mosse disponibili quella che permetter� di ottenere il punteggio migliore, attraverso il metodo \texttt{MiniMax()} della classe \texttt{Engine}. Successivamente la mossa viene mostrata, indicando con il braccio mobile prima la pedina che si vuole spostare e poi la casella in cui � stato scelto di spostarla, questo utilizzando le classi \texttt{CheckersNavigator} e \texttt{ArmController}.\\
Il livello di intelligenza della macchina � gestito mediante il livello di profondit� massimo della ricorsione, eseguita in \texttt{MiniMax()}, e viene impostato attraverso l'attributo \texttt{depth}.

\subsection{HumanPlayer}
Questa classe si differenzia dalla precedente in quanto il suo compito �
riconoscere (e non pensare) la mossa fatta dal giocatore. La ricerca avviene
analizzando tutte le possibili mosse. Per ogni mossa si controlla se la
posizione di partenza � libera o ancora occupata dalla pedina. Nel caso sia
occupata si passa a controllare la mossa successiva, altrimenti significa che
si � trovata la pedina che � stata mossa. A questo punto si fa eseguire la
ricerca della posizione in cui � stata spostata la pedina, solo tra le
destinazioni realmente possibili. Se viene riconosciuta una mossa valida, questa viene ritornata al metodo chiamante, altrimenti si lancia l'eccezione \texttt{IllegalMoveException}.

\section{Robot}
Contiene principalmente il metodo \texttt{main()} in cui si inizializzano i due
giocatori, di tipo ComputerPlayer o HumanPlayer, si fa eseguire la
calibrazione, si crea e si fa partire il gioco e si decreta il vincitore.
Questa classe istanzia anche il sensore di colore, il controllore del braccio
mobile, il \texttt{CheckersNavigator} e uno \texttt{HumanInput}\footnote{Vedi
sezione Factory} usato per l'interazione con il giocatore umano.

\section{Game}
Il metodo fondamentale di questa classe � sicuramente:
\begin{lstlisting}
public int play() throws NotCalibratedException
\end{lstlisting}
dove � gestita l'intera partita: sia l'alternarsi dei turni tra i due giocatori, sia l'aggiornamento della scacchiera virtuale memorizzata nella macchina.
\'E strutturato in un ciclo do-while che continua fino alla vittoria di uno dei giocatori. Al suo interno si controlla se il giocatore ha ancora una mossa disponibile e se l'ha effettuata correttamente, attraverso il valore restituito da \texttt{makeMove()} della classe \texttt{Player}. In caso affermativo viene aggiornata la scacchiera e si passa al giocatore successivo. In caso di mossa errata si attende per una nuova mossa corretta, altrimenti, se non vi sono pi� mosse disponibili, si decreta vincitore il giocatore avversario.

\section{Factory}
La classe \texttt{Factory} � stata implementata per garantire una scelta
univoca a livello di progetto delle diverse implementazioni disponibili per le
varie interfacce.
\begin{lstlisting}
private static CheckersNavigator navigator = MathNavigator.getInstance();
private static HumanInput HI = NXTCommHandle.getInstance();
private static ColorSensor CS = new ColorSensor(SensorPort.S1);
private static ArmController arm = ArmController.getInstance();
\end{lstlisting}
In questo modo, ad esempio, � possibile decidere di usare uno dei due\\
\texttt{CheckersNavigator} semplicemente istanziando l'oggetto che si desidera.
Cos� anche per l'input da parte del giocatore umano che pu� essere, a seconda
delle necessit�, un pulsante sul brick NXT o un segnale inviato mediante
Bluetooth.

\section{Bluetooth}
La singleton \texttt{NXTCommHandle} � una classe che implementa l'interfaccia
\texttt{HumanInput}.
Tale classe permette l'utilizzo di dispositivi remoti per la comunicazione e pi�
in particolare per il controllo del brick NXT.
Per dispositivi remoti si intendono PC o Smartphone compatibili con Java VM, con
il package \texttt{lejos.pc.comm} e che forniscono uno stack bluetooth
gestibile via \texttt{bluecove}.\\
Durante i primi test si verificava un problema durante l'inizializzazione del
Bluetooth, dovuto al costruttore della classe
\texttt{javax.bluetooth.RemoteDevice} contenuta nel \texttt{jar} di Lejos.
L'errore era dovuto al tipo di parametro atteso, una variabile di tipo
\texttt{String}. Nel codice sorgente � presente solo il costruttore avente un
argomento di tipo \texttt{byte[]}.
\begin{lstlisting}
/**
 * Note: The standard JSR 82 method for obtaining a RemoteDevice
 * uses a String rather than byte[]. Protected so shouldn't matter.
 * @param addr
 */
protected RemoteDevice(byte [] addr) { \ldots }
\end{lstlisting}
E' stato quindi effettuato un overloading, aggiungendo un nuovo costruttore
avente un parametro \texttt{String}.
Tale \texttt{RemoteDevice} carica l'indirizzo del brick NXT in formato
\texttt{byte[]} dalla classe \texttt{CommProtocol}. La classe
\texttt{CommProtocol} contiene alcune costanti come il \texttt{Bluetooth
friendly name} del brick NXT e comandi previsti dal protocollo.
\begin{lstlisting}
// Temporary FIX
// TODO Dynamically generate byte[] from String
protected RemoteDevice(String address) {
	this.addr = CommProtocol.bBT_ADDR;
		// Do Not Set Friendly name !
	this.getFriendlyName(false); // Refresh name
}
\end{lstlisting}