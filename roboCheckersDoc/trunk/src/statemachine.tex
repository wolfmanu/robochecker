\chapter{RoboCheckers Statemachine}
\section{Descrizione del progetto}
\section{Problemi di Lejos Statemachine IDE}
Durante la realizzazione della macchina a stati, si sono evidenziate una serie
di problematiche pi� o meno gravi, che hanno determinato la necessit� di
impiegare alcuni workaround. Nel seguito si illustreranno i limiti riscontrati
e le soluzioni che, ove possibile, sono state adottate.
\subsubsection{Editor grafico}
I problemi relativi all'editor sono essenzialmente legati ad una scarsa
usabilit� dell'interfaccia, ma non inficiano il corretto funzionamento del
plugin.
\begin{description} 
\item[Autocompletamento mancante] quando si modifica il codice all'interno
degli stati o degli eventi, non funzionano n� l'autocompletamento n� il
\emph{syntax checking} di \emph{Eclipse};
\item[Difficolt� di editing degli stati] durante l'editing degli stati non �
possibile andare a capo, questo riduce la leggibilit� del programma e rende
difficoltoso l'input del codice;
\item[Editing delle variabili a campi fissi] l'inserimento e la modifica di
variabili � gestita a campi fissi (probabilmente per semplificare il parsing e
la generazione del codice). Ogni variabile � definita da quattro campi
obbligatori:
\begin{lstlisting}
<modifier> <type> <name> = <value>;
\end{lstlisting}
Essendo tutti i campi obbligatori, non � possibile, ad esempio, dichiarare una
variabile non inizializzata.
\item[Cut\&Paste di variabili errato] quando si effettua il
copia/incolla di una variabile, ad esempio:
\begin{lstlisting}
private static int[] result = ...;
\end{lstlisting}
(dove \texttt{private static} � il modificatore, \texttt{int []} � il tipo,
\texttt{result} � il nome, etc\ldots), si ottiene una variabile che ha i campi
shiftati (il modificatore � \texttt{private}, il tipo � \texttt{static} il nome
� \texttt{int[] result});
\end{description}
