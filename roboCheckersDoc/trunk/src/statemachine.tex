\chapter{RoboCheckers Statemachine}
\section{Descrizione del progetto}

Lo stato iniziale conduce forzatamente allo stato \emph{Calibration}.
Una volta effettuata la calibrazione, il robot passa allo stato \emph{Home}.
In questo stato il braccio viene portato fuori dalla scacchiera.
All'evento \emph{Move}, si passa allo stato \emph{CalculateMoves}.
Questa sezione del diagramma � differente dall'implementazione sequenziale: il
tutto � gestito attraverso una serie di \emph{flag di stato}.


\includepdfmerge{img/robocheckerssmd.pdf}

\section{Osservazioni su Lejos Statemachine IDE}
Durante la realizzazione della macchina a stati, si sono evidenziate una serie
di problematiche pi� o meno gravi, che hanno determinato la necessit� di
impiegare alcuni workaround. Nel seguito si illustreranno i limiti riscontrati
e le soluzioni che, ove possibile, sono state adottate.

\subsection*{Editor grafico}
I problemi relativi all'editor sono essenzialmente legati ad una scarsa
usabilit� dell'interfaccia, ma non inficiano il corretto funzionamento del
plugin.
\begin{description} 
\item[Autocompletamento mancante] quando si modifica il codice all'interno
degli stati o degli eventi, non funzionano n� l'autocompletamento n� il
\emph{syntax checking} di \emph{Eclipse};
\item[Difficolt� di editing degli stati] durante l'editing degli stati non �
possibile andare a capo premendo invio (l'unico modo per forzare un \emph{line
break} � inserire il carattere di carriage return, ad esempio tramite
cut\&paste), questo riduce la leggibilit� del programma e rende difficoltoso
l'input del codice;
\item[Editing delle variabili a campi fissi] l'inserimento e la modifica di
variabili � gestita a campi fissi (probabilmente per semplificare il parsing e
la generazione del codice). Ogni variabile � definita da quattro campi
obbligatori:
\begin{lstlisting}
<modifier> <type> <name> = <value>;
\end{lstlisting}
Essendo tutti i campi obbligatori, non � possibile, ad esempio, dichiarare una
variabile non inizializzata;
\item[Cut\&Paste di variabili errato] quando si effettua il
copia/incolla di una variabile, ad esempio:
\begin{lstlisting}
private static int[] result = ...;
\end{lstlisting}
(dove \texttt{private static} � il modificatore, \texttt{int []} � il tipo,
\texttt{result} � il nome, etc\ldots), si ottiene una variabile che ha i campi
shiftati (il modificatore � \texttt{private}, il tipo � \texttt{static} il nome
� \texttt{int[] result});
\end{description}

\subsection*{Lista statica degli import}
Il generatore di codice non consente all'utente di inserire statement
arbritrario prima della dichiarazione della classe che contiene il resto del
programma (ad esempio per modificare la lista degli \texttt{import}).\\
Questo limite ha reso indispensabile modificare parte del plugin, in modo da
poter aggiungere gli \texttt{import} dei package contenti le classi helper
necessarie alla navigazione, alla gestione della partita a Dama, etc\ldots~
Nello specifico si � modificato il template di traduzione contenuto nel file
\texttt{template/statemachine.xpt} (si veda Appendice \ref{sec:template}).

\subsection*{Gestione degli eventi}
\emph{Lejos} fornisce un approccio piuttosto limitato alla programmazione ad
eventi. L'API relativa � denominata \texttt{Behavior} e permette di definire un
certo numero di ``comportamenti'' (stati) da assegnare al robot.\\
La transizione tra gli stati � affidata ad un'istanza della classe
\texttt{Arbitrator}, che si occupa di effettuare periodicamente un polling su
tutti gli stati per controllare se sono scattati o meno gli eventi che
determinano una transizione. Gli eventi, pertanto, non sono veri e propri
interrupt asincroni, ma il risultato della valutazione (eseguita centinaia di
volte al secondo) di un'espressione booleana definita dall'utente.\\
\emph{Lejos Statemachine} implementa un meccanismo molto simile (probabilmente
un wrapper evoluto di \texttt{Behavior}), che permette di definire due tipi di
transizioni\footnote{Si veda Appendice \ref{sec:howtouse} per maggiori
dettagli.}:
\begin{description}
\item[Transizione mediata da evento] o forzante, transisce appena si verifica
la condizione definita nell'evento. Si realizza interponendo un oggetto
\texttt{Event} tra due stati;
\item[Transizione standard] attende che l'esecuzione del codice contenuto nello
stato sia terminata e transisce solo se la condizione associata � verificata.
\end{description}
La gestione degli eventi risulta essere quindi una sorta di ``simulazione'' che
poco si addice alla natura di una macchina a stati.

\subsection*{Occupazione di memoria}
\begin{quote}
``\small
\textit{Nobody will ever need more than 64K} 
\normalsize''\footnote{B. Gates, \textit{The future of computing}, 1981}
\end{quote}
