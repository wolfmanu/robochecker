\frontespizio
\indici
\sommario
Il progetto consiste in un robot, realizzato con 
Lego\textregistered~Mindstorms, 
che giochi a dama contro un avversario umano.\\
\\
Esistono due tipi di Lego\textregistered~Mindstorms: l'RCX, pi� vecchio,
e l'NXT.\\
Le differenze sostanziali sono nel ``mattoncino'' intelligente (detto
\emph{brick}): NXT consente di utilizzare sensori pi� evoluti
e motori pi� precisi rispetto ad RCX.\\
Per entrambi Lego\textregistered~commercializza degli ambienti
di sviluppo per realizzare il software da caricare sui brick.\\
Per maggiori dettagli: \url{http://mindstorms.lego.com}.\\
\\
Nel progetto, denominato \emph{RoboCheckers}, � stato utilizzato
il Lego\textregistered~Mindstorms NXT. Vista la complessit� 
non banale, si � deciso di utilizzare un linguaggio 
di programmazione open-source derivato da Java, chiamato \emph{Lejos}.\\
\emph{Lejos} permette di scrivere codice da compilare che deve
poi essere caricato sul brick tramite un'apposita utility.\\
� possibile scaricare dal sito web di \emph{Lejos} 
(\url{http://lejos.sourceforge.net}) un plugin
da integrare nell'ambiente di sviluppo \emph{Eclipse}
(\url{http://www.eclipse.org}), che agevola la programmazione. 