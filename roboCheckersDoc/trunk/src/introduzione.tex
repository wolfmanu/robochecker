\frontespizio
\indici
\sommario
Il progetto consiste di un Robot, realizzato con Lego\textregistered~Mindstorms, 
in grado di disputare una partita a Dama contro un avversario umano.\\
Esistono due tipi di Lego\textregistered~Mindstorms: la versione RCX, pi�
vecchia, e quella NXT.\\
Le differenze sostanziali sono nel ``mattoncino'' intelligente (detto
\emph{brick}): NXT consente di utilizzare sensori pi� evoluti
e motori pi� precisi rispetto ad RCX.\\
Per entrambi, Lego\textregistered~commercializza degli ambienti
di sviluppo per realizzare il software da caricare sui brick.\\
Per maggiori dettagli: \url{http://mindstorms.lego.com}.
Il progetto � basato sui Lego\textregistered~Mindstorms NXT.

\paragraph{}
Dal punto di vista software, considerata la complessit� non banale di ci� che
si intendeva realizzare, si � deciso di utilizzare un linguaggio di
programmazione open-source derivato da Java, chiamato \emph{Lejos}.\\
\emph{Lejos} permette di scrivere codice sorgente Java-like, che
deve poi essere compilato e caricato, tramite apposite utility, sul brick, dove
viene interpretato dalla \emph{Lejos Virtual Machine} inclusa nel firmware
\emph{Lejos}.\\ � possibile scaricare dal sito web \url{http://lejos.sourceforge.net}
un plugin da integrare nell'ambiente di sviluppo \emph{Eclipse}
(\url{http://www.eclipse.org}), che agevola l'intero processo di programmazione.

\paragraph{}
Come secondo obiettivo, si � scelto di implementare nel Robot il medesimo
comportamento ottenuto mediante programmazione di tipo tradizionale, servendosi
invece di un plugin (tutt'ora in fase di sviluppo) denominato \emph{Lejos Statemachine 
Development Toolkit}, che consente di programmare secondo un modello di
macchina a stati.\\
Maggiori informazioni a riguardo saranno fornite nel seguito e in appendice.\\
Per raffrontare quanto si trover� scritto pi� avanti con il codice allegato a
questa relazione, si tenga presente che la parte di progetto che poggia
direttamente su \emph{Lejos} � denominata \texttt{RoboCheckersNXJ}, mentre
quella implementata con la \emph{Statemachine} si chiama \texttt{RoboCheckers}.
