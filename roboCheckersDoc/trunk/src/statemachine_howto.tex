\appendix
\chapter{Lejos Statemachine}
\emph{Lejos Statemachine development toolkit} � una utility per l'ambiente di
sviluppo \emph{Eclipse}, che consente di progettare graficamente una macchina a
stati e implementarla in un brick Lego\textregistered~NXT. Il plugin genera codice \emph{Lejos}, pertanto
consente al programmatore di inserire normale statement Java (comprendendo, chiaramente,
le API di Lejos) all'interno del diagramma a stati.\\
Il manuale ufficiale (in lingua tedesca) del plugin � disponibile al seguente
indirizzo: \url{http://fermat.nordakademie.de/update/Beschreibung.pdf};
l'autore Frank Zimmermann, ne ha prodotto anche un breve riassunto in inglese
che si pu� trovare qui: \url{http://lejos.sourceforge.net/forum/viewtopic.php?t=675}.\\
Una guida completa all'installazione di \emph{Lejos} e di \emph{Lejos
Statemachine development toolkit} � stata scritta in inglese da Juan Antonio
Bre�a Moral e si pu� trovare a questo indirizzo\footnote{I documenti citati in
questo paragrafo sono allegati alla presente relazione.}:
\url{http://www.juanantonio.info/p_articles/archive/2008/LEJOS-NXJ-EBOOK.pdf}
(Capitoli 2.2, 2.3, 3.1).

\section{Ricompilazione del plugin}
La versione ufficiale di \emph{Lejos Statemachine} presente a questo link:
\url{http://fermat.nordakademie.de/update/} e installabile seguendo i
riferimenti forniti in precedenza, � basata su una vecchia versione delle API di
\emph{Lejos}.\\
La release del progetto \emph{Lejos NXJ} pi� recente al momento attuale
(Febbraio 2009) � la \texttt{0.7.0}, se si vuole avere compatibilit� con questa
versione delle API � pertanto necessario ottenere il sorgente (attualmente in
fase di sviluppo) del plugin e compilarlo.

\subsection{Checkout dal repository SVN}
Per prima cosa occorre disporre un ambiente Eclipse provvisto di
\emph{Plug-in Development Environment}, ad esempio la versione \emph{Classic}
disponibile a questo indirizzo: \url{http://www.eclipse.org/downloads/}.\\
Per eseguire il checkout del progetto � necessario installare \emph{Subclipse},
un plugin che integra \emph{Subversion} (Sezione \ref{sec:subversion}) in
\emph{Eclipse}. Per farlo, selezionare il men�
\texttt{Help}$\to$\texttt{Software Updates} e nella sezione \emph{Available Software} selezionare
\texttt{Add Site}.\\
	\begin{figure}
		\begin{center}
			\includegraphics[scale=0.45]{img/howto01.png}
			\caption{Installazione di Subclipse \label{fig:howto01}}
		\end{center}
	\end{figure}
\paragraph{}
Nella finestra di dialogo digitare
\url{http://subclipse.tigris.org/update_1.4.x} e premere \texttt{OK}.\\
Espandere la voce appena aggiunta e selezionare tutti i sotto-elementi (o
almeno quelli evidenziati come \texttt{required} o \texttt{recommended}),
successivamente premere \texttt{Install} e procedere con l'installazione.
	\begin{figure}
		\begin{center}
			\includegraphics[scale=0.9]{img/howto02.png}
			\caption{Installazione di Subclipse \label{fig:howto02}}
		\end{center}
	\end{figure}
\paragraph{}
Una volta riavviato Eclipse, procedere all'import del progetto in questo modo:
selezionare \texttt{Files}$\to$\texttt{New}$\to$\texttt{Other} e, nella
finestra di dialogo, \texttt{SVN}$\to$\texttt{Checkout projects from SVN},
premere quindi \texttt{Next}.\\
Nella successiva finestra di dialogo selezionare \texttt{Create a new
repository location} e premere \texttt{Next}.
	\begin{figure}
		\begin{center}
			\includegraphics[scale=0.7]{img/howto03.png}
			\caption{Aggiunta del repository\label{fig:howto03}}
		\end{center}
	\end{figure}
\paragraph{}
Digitare l'indirizzo \url{http://svn2.assembla.com/svn/vldt/development/} e
premere ancora \texttt{Next}.\\
	\begin{figure}
		\begin{center}
			\includegraphics[scale=0.7]{img/howto04.png}
			\caption{Inserimento dell'url del repository SVN \label{fig:howto04}}
		\end{center}
	\end{figure}
Nella finestra successiva selezionare l'intera directory SVN (i.e. tutti i
progetti presenti) e premere \texttt{Finish}.\\
A questo punto il workspace dovrebbe presentarsi come in Figura
\ref{fig:howto05}
	\begin{figure}
		\begin{center}
			\includegraphics[scale=0.8]{img/howto05.png}
			\caption{Stato finale del workspace\label{fig:howto05}}
		\end{center}
	\end{figure}

\subsection{Modifica del template}
\label{sec:template}
Il codice \texttt{Lejos} generato dal plugin si basa, come gi� evidenziato in
precedenza, sulle API standard, ma anche su una serie di classi facenti parte
del plugin stesso che implementano l'intero meccanismo della macchina a stati,
le transizioni, gli eventi etc\dots\\
Per questo motivo � essenziale che tutti questi helper vengano importati in
testa al programma generato dal parser grafico, infatti il seguente segmento di
codice � sempre presente quando si compila una \emph{Statemachine}:
\begin{lstlisting}
	import de.nordakademie.lejos.statemachine.*;
	import lejos.nxt.*;
	import lejos.navigation.*;
\end{lstlisting}
Questa lista di import � statica e non direttamente accessibile
all'utilizzatore\footnote{A meno di scompattare i files .jar del plugin e
cercare al loro interno il template di traduzione.}, tuttavia a questo livello
� possibile modificarla a piacimento, semplicemente andando ad editare il
template usato per la traduzione in codice \emph{Lejos}. Il file in questione
si trova in \texttt{de.nordakademie.lejos.core/src/template/\\statemachine.xpt},
per il progetto roboCheckers lo si � modificato come segue:
\begin{lstlisting}
[...]
 �EXPAND javaClass(btFlag) FOREACH states.typeSelect(StateMachine)�
 �FILE name+".java"�
    
	import de.nordakademie.lejos.statemachine.*;
	import lejos.nxt.*;
	import lejos.navigation.*;
	import it.polito.Navigation.*;
	import it.polito.util.*;
	import it.polito.Checkers.*;
	import lejos.nxt.addon.ColorSensor;
	import it.polito.BluetoothComm.*;

     public class �name� extends Statemachine{
     
        public �name�(){this(null);} 
[...]
\end{lstlisting}
\newpage
\subsection{Generazione del plugin}
Per esportare il plugin selezionare \texttt{File}$\to$\texttt{Export} e
successivamente \texttt{Plug-in Development}$\to$\texttt{Deployable features},
premere quindi su \texttt{Next}.\\
	\begin{figure}
		\begin{center}
			\includegraphics[scale=0.6]{img/howto06.png}
			\caption{Men� Export\label{fig:howto06}}
		\end{center}
	\end{figure}
	\begin{figure}
		\begin{center}
			\includegraphics[scale=0.6]{img/howto07.png}
			\caption{Deployable features\label{fig:howto07}}
		\end{center}
	\end{figure}
Nella finestra successiva, selezionare la voce \texttt{de.nordakademie.lejos} e
scegliere una directory di destinazione, prestare attenzione a impostare la
scheda \texttt{Options} come in Figura \ref{fig:howto08}.\\
	\begin{figure}
		\begin{center}
			\includegraphics[scale=0.6]{img/howto08.png}
			\caption{Deployable features, options\label{fig:howto08}}
		\end{center}
	\end{figure}
Premere quindi su \texttt{Finish}.\\
\paragraph{}
A questo punto non resta che installare la feature appena generata attraverso
la procedura gi� illustrata per installare i plugin, avendo cura di specificare
la directory locale scelta in precedenza, come repository sorgente nella
finestra di dialogo \texttt{Add Site}.\\
Per evitare conflitti, si consiglia di disinstallare preventivamente qualsiasi
versione del plugin presente in \emph{Eclipse}.

\section{Uso di Lejos Statemachine} 

