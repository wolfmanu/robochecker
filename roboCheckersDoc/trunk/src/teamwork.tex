\chapter{Team Work}
Tra i vari progetti sviluppati in passato non ci si era mai posti il problema
di come gestire il lavoro di gruppo in quanto nessuno dei precedenti progetti
richiedeva la scrittura di circa 2000 righe di codice. Vista l'esperienza
maturata esternamente al Politecnico da alcuni componenti del gruppo � stato
proposto di utilizzare un sistema di controllo versioni, cos� da gestire
facilmente la scrittura simultanea del codice da parte di pi� persone. Tra i
vari sistemi disponibile � stato scelto Subversion principalmente perch� quello
conosciuto da pi� componenti del gruppo.
\section{Subverion}
Subverion (SVN) ed in generale qualunque programma per il controllo versione
viene utilizzato nel mondo ingegneristico e in particolare in quello informatico per
esempio la possibilit� di tornare ad uno stato precedente del progetto, nei
casi in cui si raggiungeva un vicolo cieco.
SVN permette di tenere traccia e di controllare i cambiamenti al codice
sorgente, sapendo sempre chi � l'autore della singola modifica. 
Man mano che il software viene sviluppato, � sempre pi� probabile che versioni distinte 
dello stesso software siano dispiegate in posti diversi, e che gli sviluppatori
del software lavorando autonomamente allo sviluppo rischino di avere
successivamente difficolta a condividere con gli altri membri del team le
proprie modifiche.
Tradizionalmente, i sistemi di controllo versione hanno usato un modello
centralizzato, in cui tutte le funzioni di controllo versione sono eseguite da
un server condiviso. Alcuni anni fa, certi sistemi come TeamWare, BitKeeper e
GNU arch hanno cominciato a usare un modello distribuito, in cui ogni
sviluppatore lavora direttamente con il suo repository locale, e le modifiche
sono condivise tra i repository in un passo separato. Questa modalit� di
operare permette di lavorare senza una connessione di rete, e consente anche
agli sviluppatori di accedere alle funzioni di controllo versione senza aver
bisogno di permessi concessi da un'autorit� centrale.
Nella maggior parte dei progetti di sviluppo software, come il caso di questo
``piccolo'' progetto, pi� sviluppatori lavorano in parallelo sullo stesso
software. Se due sviluppatori tentano di modificare lo stesso file
contemporaneamente, in assenza di un metodo di gestione degli accessi, essi
possono facilmente sovrascrivere o perdere le modifiche effettuate contestualmente.
Alcuni sistemi prevengono i problemi dovuti ad accessi simultanei,
semplicemente bloccando (lock) i file, cosicch� solamente uno sviluppatore per
volta ha diritto di accesso in scrittura alla copia di quel file contenuta nel
repository centrale. Altri, come quello scelto per il nostro progetto,
permettono a pi� sviluppatori di modificare lo stesso file nello stesso tempo,
e forniscono degli strumenti per combinare le modifiche in seguito (merge).
La maggior parte dei sistemi di controllo versione usano la compressione delta,
che conserva solamente le differenze fra le versioni successive dei file.
Questo consente un immagazzinamento efficiente di pi� versioni di un file,
purch�, come solitamente succede, le modifiche tra una versione e la successiva
riguardino solamente una piccola parte del testo.

\section{Glossario di Subverion}
Di seguito elenchiamo i pricipali termini con cui in questi due mesi siamo
diventati famigliari per utilizzare Subverion:

\begin{description} 
\item[Repository]
    Il repository � dove i file sono memorizzati, spesso su un
    server\footnote{Nel nostro caso il server � stato gentilmente ospitato sul
    server di un membro del gruppo} esposto su internet.
\item[Commit] 
    Un commit (o, pi� raramente, install, submit o check-in) si effettua quando
    si copiano le modifiche fatte su file locali nella directory (il software
    di controllo versione controlla quali file sono stati modificati
    dall'ultima sincronizzazione). 
\item[Check-Out]
    Un check-out (o checkout o co) effettua una copia di lavoro dal repository
    (pu� essere visto come l'operazione contraria all'importazione).
\item[Update]
    Un update (o sync) copia le modifiche fatte sul repository nella propria
    directory di lavoro (pu� essere visto come l'operazione contraria al commit).
\item[Merge / Integrazione]
    Un merge o integrazione unisce modifiche concorrenti in una revisione unificata.
\item[Revisione] 
    Una revisione o versione � una versione in una catena di modifiche.
\item[Conflitto] 
    Un conflitto si presenta quando diversi soggetti fanno modifiche allo
    stesso documento. Non essendo il software abbastanza intelligente da
    decidere quale tra le modifiche � quella 'corretta', si richiede ad un
    utente di risolvere il conflitto. Risolvere L'intervento di un utente per
    la risoluzione di un conflitto tra modifiche differenti di uno stesso documento.
\end{description}
