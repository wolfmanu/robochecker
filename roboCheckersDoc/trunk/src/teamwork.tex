\chapter{Team Work}
Tra i vari progetti sviluppati in passato non ci si era mai posti il problema
di come gestire il lavoro di gruppo. Nessun precedente lavoro di gruppo fatto
al politecnico richiedeva la scrittura di circa 2000 righe di codice. Quindi ci
siamo fin da subito posti il problema di coordinarci e sincronizzare i lavori
fatti da ognuno. Vista l'esperienza maturata esternamente al Politecnico da
alcuni componenti del gruppo � stato proposto di utilizzare un sistema di
controllo versioni, cos� da gestire facilmente la scrittura simultanea del
codice da parte di pi� persone.

Tra i vari sistemi disponibili � stato scelto Subversion principalmente perch�
quello conosciuto da pi� componenti del gruppo.
\section{Subversion}
\label{sec:subversion}
Subverion o pi� brevemente SVN, ed in generale qualunque programma per il
controllo versione, viene utilizzato nel mondo ingegneristico e in particolare
nel campo informatico al fine di semplificare le operazioni per gestire un
lavoro di gruppo. Tra le funzioni svolte da SVN troviamo la possibilit� di
tornare ad uno stato precedente del progetto, tenere traccia dei
cambiamenti al codice e sincronizzare il lavoro svolto dai membri del gruppo.

Tutte queste funzioni sono risultate molto utile nello scrivere il progetto,
infatti spesso ci si trova ad aver intrapreso delle strade che non si sono
dimostrate praticabili, quindi con poche operazioni si � pututo facilmente
ritornare al codice funzionante senza doversi ricordare tutte le modifiche
fatte. 

Molto utile � anche stata la funzione di sincronizzazione che ci ha permesso di
lavorare sullo stesso codice da posti diversi dalle nostre case ai computer dei
laboratori il tutto senza perdere la sincronizzazione con il lavoro fatto dagli
altri membri del gruppo, questo sia per la scrittura del codice Java che per la
scrittura dei questa relazione in \LaTeX.

Tradizionalmente, i sistemi di controllo versione avevano usato un modello
centralizzato, in cui tutte le funzioni di controllo versione erano eseguite da
un server condiviso. Alcuni anni fa, certi sistemi come TeamWare, BitKeeper e
GNU arch hanno cominciato a usare un modello distribuito, in cui ogni
sviluppatore lavora direttamente con il suo repository locale, e le modifiche 
sono condivise tra i repository in un passo separato. Questa modalit� di
operare permette di lavorare senza una connessione di rete, e consente anche
agli sviluppatori di accedere alle funzioni di controllo versione senza aver
bisogno di permessi concessi da un'autorit� centrale.

Nella maggior parte dei progetti di sviluppo software, come il caso di questo
``piccolo'' progetto, pi� sviluppatori lavorano in parallelo sullo stesso
software. Se due sviluppatori tentano di modificare lo stesso file
contemporaneamente, in assenza di un metodo di gestione degli accessi, essi
possono facilmente sovrascrivere o perdere le modifiche effettuate
contestualmente. Alcuni sistemi prevengono i problemi dovuti ad accessi
simultanei, semplicemente bloccando (lock) i file, cosicch� solamente uno
sviluppatore per volta ha diritto di accesso in scrittura alla copia di quel
file contenuta nel repository centrale. Altri, come quello scelto per il nostro
progetto, permettono a pi� sviluppatori di modificare lo stesso file nello
stesso tempo, e forniscono degli strumenti per combinare le modifiche in
seguito (merge).

La maggior parte dei sistemi di controllo versione usano la
compressione delta, che conserva solamente le differenze fra le versioni
successive dei file. Questo consente un immagazzinamento efficiente di pi�
versioni di un file, purch�, come solitamente succede, le modifiche tra una
versione e la successiva riguardino solamente una piccola parte del testo.

\section{Glossario di Subverion}
Di seguito elenchiamo i principali termini con cui in questi due mesi siamo
diventati famigliari per utilizzare Subverion:

\begin{description} 
\item[Repository]
    Il repository � dove i file sono memorizzati, spesso su un
    server\footnote{Nel nostro caso il server � stato gentilmente ospitato sul
    server di un membro del gruppo} esposto su internet.
\item[Commit] 
    Un commit (o, pi� raramente, install, submit o check-in) si effettua quando
    si copiano le modifiche fatte su file locali nella directory (il software
    di controllo versione controlla quali file sono stati modificati
    dall'ultima sincronizzazione). 
\item[Check-Out]
    Un check-out (o checkout o co) effettua una copia di lavoro dal repository
    (pu� essere visto come l'operazione contraria all'importazione).
\item[Update]
    Un update (o sync) copia le modifiche fatte sul repository nella propria
    directory di lavoro (pu� essere visto come l'operazione contraria al commit).
\item[Merge / Integrazione]
    Un merge o integrazione unisce modifiche concorrenti in una revisione unificata.
\item[Revisione] 
    Una revisione o versione � una versione in una catena di modifiche.
\item[Conflitto] 
    Un conflitto si presenta quando diversi soggetti fanno modifiche allo
    stesso documento. Non essendo il software abbastanza intelligente da
    decidere quale tra le modifiche � quella 'corretta', si richiede ad un
    utente di risolvere il conflitto. Risolvere L'intervento di un utente per
    la risoluzione di un conflitto tra modifiche differenti di uno stesso documento.
\end{description}
