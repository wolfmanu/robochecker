\chapter{Meccanica del Robot}
Il robot � costituito da tre diverse parti funzionali: una base dotata di ruote motrici,
 un braccio rotante ed un sistema di leve a forbice.

\section{La base}
La base del robot � stata costruita partendo da un modello presente sul sito
della Lego 
(\emph{http://www.active-robots.com/products/mindstorms4schools/building-instructions/Build-RoboArm.pdf} Figura \ref{Base}) e modificato come segue: il motore B che nello schema
originale si occupava di far ruotare il braccio continua nella sua funzione mentre il motore A trasferisce la sua funzione da
far muovere in avanti il braccio a gestire la trazione del robot.\\

Per permettere al robot di avanzare la base � stata montata su quattro ruote di
cui le due anteriori sono le ruote motrici collegate tra di loro con un asse
motore in comune che riceve il movimento tramite un gioco di ingranaggi conici
che permettono di trasformare la rotazione dall'asse veticale prodotta dal
motore a pi� la vite archimedea in un movimento sull'asse orizzontale.
 
 
 \begin{figure}[htbp]
\begin{center}
\includegraphics[scale=0.5]{img/base.png}
\caption{Struttura iniziale della Base \label{Base}}
\end{center}
\end{figure}